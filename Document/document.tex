\documentclass[12pt, a4paper]{scrartcl}

\usepackage{xcolor}
\usepackage{listings}
\usepackage{etoolbox}
\usepackage{color}

\definecolor{base0}{RGB}{131,148,150}
\definecolor{base01}{RGB}{88,110,117}
\definecolor{base2}{RGB}{238,232,213}
\definecolor{sgreen}{RGB}{133,153,0}
\definecolor{sblue}{RGB}{38,138,210}
\definecolor{scyan}{RGB}{42,161,151}
\definecolor{smagenta}{RGB}{211,54,130}


\newcommand\digitstyle{\color{smagenta}}
\newcommand\symbolstyle{\color{base01}}
\makeatletter
\newcommand{\ProcessDigit}[1]
{%
  \ifnum\lst@mode=\lst@Pmode\relax%
   {\digitstyle #1}%
  \else
    #1%
  \fi
}
\makeatother


\lstdefinestyle{solarizedcsharp} {
  language=[Sharp]C,
  frame=lr,
  linewidth=160mm,
  breaklines=true,
  tabsize=2,
  numbers=left,
  numbersep=5pt,
  firstnumber=auto,
  numberstyle=\tiny\ttfamily\color{base0},
  rulecolor=\color{base2},
  basicstyle=\footnotesize\ttfamily,
  commentstyle=\color{base01},
  morecomment=[s][\color{base01}]{/*+}{*/},
  morecomment=[s][\color{base01}]{/*-}{*/},
  morekeywords={  abstract, event, new, struct,
                as, explicit, null, switch,
                base, extern, object, this,
                bool, false, operator, throw,
                break, finally, out, true,
                byte, fixed, override, try,
                case, float, params, typeof,
                catch, for, private, uint,
                char, foreach, protected, ulong,
                checked, goto, public, unchecked,
                class, if, readonly, unsafe,
                const, implicit, ref, ushort,
                continue, in, return, using,
                decimal, int, sbyte, virtual,
                default, interface, sealed, volatile,
                delegate, internal, short, void,
                do, is, sizeof, while,
                double, lock, stackalloc,
                else, long, static,
                enum, namespace, string, var},
  keywordstyle=\bfseries\color{sgreen},
  showstringspaces=false,
  stringstyle=\color{scyan},
  identifierstyle=\color{sblue},
  extendedchars=true,
  literate=
    {0}{{{\ProcessDigit{0}}}}1
    {1}{{{\ProcessDigit{1}}}}1
    {2}{{{\ProcessDigit{2}}}}1
    {3}{{{\ProcessDigit{3}}}}1
    {4}{{{\ProcessDigit{4}}}}1
    {5}{{{\ProcessDigit{5}}}}1
    {6}{{{\ProcessDigit{6}}}}1
    {7}{{{\ProcessDigit{7}}}}1
    {8}{{{\ProcessDigit{8}}}}1
    {9}{{{\ProcessDigit{9}}}}1
    {\}}{{\symbolstyle{\}}}}1
    {\{}{{\symbolstyle{\{}}}1
    {(}{{\symbolstyle{(}}}1
    {)}{{\symbolstyle{)}}}1
    {=}{{\symbolstyle{$=$}}}1
    {;}{{\symbolstyle{$;$}}}1
    {>}{{\symbolstyle{$>$}}}1
    {<}{{\symbolstyle{$<$}}}1
    {\%}{{\symbolstyle{$\%$}}}1,
}

\lstset{escapechar=@,style=solarizedcsharp}

\begin{document}
\hyphenpenalty=10000
\exhyphenpenalty=10000


\section{ILogger interface}
I implemented an ILogger interface that contains a Log(string) function.
Than, I created the ConsoleLogger class that inherits the ILogger interface.
This way, we apply Liskov's Substitution Principle, because
we can always use another class that inherits ILogger, to 
print all the messages in another environment.
The Interface Segregation Principle is also applied, because 
we have a small interface with a function that we use.

I created the \_logger variable inside the HotelReception class.
The \_logger variable is of type ILogger, this way we apply the 
Dependency Inversion Principle.
This variable is private for a better encapsulation.
Than I added the HotelReception constructor
and I'm using it to assign a reference to \_logger.

I put the ILogger interface and the Logger class into a separate
folder for a better structure of the project.

\section{File reading}
I created an interface IFileReader with a function Read(). Than 
I made a FileReader class that inherits the interface. The 
Read() function from the FileReader class returns all the 
content from the json file. This class has, also, a constructor 
for assigning a reference to a \_logger variable. 

I decided to do that, because 
the Single Responsibility Principle was violated, since the 
reading of the text file should be in its own class.
Than I created the \_fileReader variable of type IFileReader, 
inside the HotelReception 
class. This way, we apply the Dependency Inversion Principle,
because the dependency is of interface type, and, also we apply
Liskov's Substitution Principle, 
because we can always use another class that inherits IFileReader.
This variable is private for a better encapsulation.

In the constructor of the HotelReception class I assigned a 
reference to this variable.

In the ProcessOrder() function I used \_fileReader.Read() to get 
the content of the json file.

I put the IFileReader interface and the FileReader class into a 
separate folder for a better structure of the project.

\section{Order class to OrderData}

I changed the Order class to an OrderData class and I'll read the data 
from the json file into an object of type OrderData.

\section{OrderSerializer}

I created an interface IOrderSerializer with a function Deserialize(). Than 
I made an OrderSerializer class that inherits the interface. The 
Deserialize() function from the OrderSerializer class returns the 
OrderData from the json string. This class has, also, a constructor 
for assigning a reference to a \_logger variable. 
I decided to do that, because 
the Single Responsibility Principle was violated, since the 
deserialization of the orderJson should be in its own class.

Than I created the \_orderSerializer variable of type IOrderSerializer, 
inside the HotelReception 
class. This way, we apply the Dependency Inversion Principle,
because the dependency is of interface type, and, also we apply
Liskov's Substitution Principle, 
because we can always use another class that inherits IOrderSerializer.
That variable is private for a better encapsulation.
In the constructor of the HotelReception class I assigned a 
reference to that variable.

In the ProcessOrder() function I used \_orderSerializer.Deserialize() 
to deserialize the OrderData.

I put the IOrderSerializer interface and the OrderSerializer class into a 
separate folder for a better structure of the project.

\section{Orders}

I made an Order abstract class as a base class for all the orders. This way,
the project will follow the abstraction principle. I created 
an ILogger variable inside that class. That variable is protected, because 
it will be used in inherited classes. I also defined an abstract Process(OrderData) function.
In the inherited classes, that function will calculate the Final Price and will return it.
If the order was not processed, the function will return 0. 

In Order class I defined an abstract function GetStringType() witch should return for every 
inherited class its type, as a string. This function is helpfull, because we observe that,
in all the 3 order types, we have some validation for the data of the order. So, I created 
a function for each validation, because the logs were almost the same for all order types. 
This way, the code in the Process(OrderData) function is smaller,
more visible and it's not repeated in all the inherited classes of the Order class.

Another good advantage of this method is 
that we avoid mistakes, like writing "Room" instead of "Product" and log the wrong message 
to the screen. Also, if in the future we want to add new orders, we can use those functions 
for them too and if those orders have some new properties, we can add new functions 
to validate them, as well. So, this way we apply the Open/Closed Principle.

I created the RoomOrder class for the room type of an order. In the Process(OrderData) function 
I stored the orderData.Quantity * orderData.Price into a variable, because its calculation
appears multiple times and the code is repetitive. 

For the ProductOrder and the BreakfastOrder classes, I made the same chances as in the RoomOrder. 

This way, we apply the Single Responsibility Principle, because the HotelReception class
shouldn't deal with the processing of each individual order. 

\section{Unknown Order}

We need an Unknown order type for the default case of the switch. So, I added Unknown to the enum,
and I created the UnknownOrder class, that inherits the Order class. The Process function
of this class only returns 0.

I changed the Deserialize() function like this:

\begin{lstlisting}
public OrderData DeserializeOrder(string orderJson)
{
    _logger.Log("Deserializing Order object from json data...");
    OrderData orderData = JsonConvert.DeserializeObject<OrderData>(
        orderJson, new StringEnumConverter());
    
    if (orderData == null) 
    { 
        orderData = new OrderData();
        orderData.Type = OrderType.Unknown;
    }
    
    return orderData;
}
\end{lstlisting}

The switch from the ProcessOrder() function was moved to a new class, OrderFactory, which 
inherits an IOrderFactory interface. This interface contains a Create(OrderData orderData)
function that return an object of type Order. 

\end{document}